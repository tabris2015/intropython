\documentclass[10pt]{beamer}

\usepackage[spanish, mexico]{babel}
\usepackage[utf8]{inputenc}

\usetheme[progressbar=frametitle]{metropolis}
\usepackage{appendixnumberbeamer}

\usepackage{booktabs}
\usepackage[scale=2]{ccicons}

\usepackage{pgfplots}
\usepgfplotslibrary{dateplot}

\usepackage{xspace}
\newcommand{\themename}{\textbf{\textsc{metropolis}}\xspace}

%%
\usepackage{color}
\definecolor{lstgrey}{rgb}{0.95,0.95,0.95}
\definecolor{mygreen}{RGB}{28,172,0} % color values Red, Green, Blue
\definecolor{mylilas}{RGB}{170,55,241}

\usepackage{listings}
\lstset{language=Python,
       backgroundcolor=\color{lstgrey},
       frame=single,
       basicstyle=\footnotesize\ttfamily,
       captionpos=b,
       tabsize=2,
  }

\lstset{language=Python,%
  %basicstyle=\color{red},
  breaklines=true,%
  morekeywords={python2tikz},
  keywordstyle=\color{blue},%
  morekeywords=[2]{1}, keywordstyle=[2]{\color{black}},
  identifierstyle=\color{black},%
  stringstyle=\color{mylilas},
  commentstyle=\color{mygreen},%
  showstringspaces=false,%without this there will be a symbol in the places where there is a space
  numbers=left,%
  numberstyle={\tiny \color{black}},% size of the numbers
  numbersep=9pt, % this defines how far the numbers are from the text
  emph=[1]{for,end,break},emphstyle=[1]\color{red}, %some words to emphasise
  %emph=[2]{word1,word2}, emphstyle=[2]{style},    
}
%

\lstset{language=C,
       backgroundcolor=\color{lstgrey},
       frame=single,
       basicstyle=\footnotesize\ttfamily,
       captionpos=b,
       tabsize=2,
  }

\lstset{language=C,%
  %basicstyle=\color{red},
  breaklines=true,%
  morekeywords={c2tikz},
  keywordstyle=\color{blue},%
  morekeywords=[2]{1}, keywordstyle=[2]{\color{black}},
  identifierstyle=\color{black},%
  stringstyle=\color{mylilas},
  commentstyle=\color{mygreen},%
  showstringspaces=false,%without this there will be a symbol in the places where there is a space
  numbers=left,%
  numberstyle={\tiny \color{black}},% size of the numbers
  numbersep=9pt, % this defines how far the numbers are from the text
  emph=[1]{for,end,break},emphstyle=[1]\color{red}, %some words to emphasise
  %emph=[2]{word1,word2}, emphstyle=[2]{style},    
}
%


\title{Introducción a la programación con Python 3}
\subtitle{Introducción y generalidades}
\date{\today}
% \date{}
\author{Ing. Jose Eduardo Laruta Espejo}
\institute{Cognos}
% \titlegraphic{\hfill\includegraphics[height=1.5cm]{logo.pdf}}

\begin{document}

\maketitle

\begin{frame}[allowframebreaks]{Contenido}
  \setbeamertemplate{section in toc}[sections numbered]
  \tableofcontents[]
\end{frame}

%%%

\section{Introducción a Python3}
\subsection{Historia y características}

\begin{frame}{Python}
    \begin{columns}
        \begin{column}{0.6\textwidth}
            Es un lenguaje de programación interpretado que tiene como filosofía la simplicidad y 
            legibilidad del código.
            
            Actualmente es uno de los lenguajes más populares según distintos rankings.
            
            Existen aplicaciones en todas las areas de la tecnología usando python.

        \end{column}
        \begin{column}{0.4\textwidth}
            \begin{figure}[!h] 
                \centering
                \includegraphics[width=0.95\textwidth]{img/logo}
                % \caption[Algunas aplicaciones de la visión artificial]{Algunas aplicaciones de la visión artificial. Fuente: \cite{Szeliski2011} }
            \end{figure}
        \end{column}
    \end{columns}
    
\end{frame}

\begin{frame}{Historia}
    \begin{columns}
        \begin{column}{0.6\textwidth}
            \begin{itemize}
                \item La primera versión pública (0.9) fue publicada en 1991.
                \item La versión 1.0 fue publicada en 1994.
                \item La versión 2.0 se lanzó en 2000.
                \item La versión 3.0 se lanza en 2009.
            \end{itemize}
            Actualmente, la versión 2.7 ha sido discontinuada y la versión 3.5+ es la recomendada 
            para nuevos desarrollos.
        \end{column}
        \begin{column}{0.4\textwidth}
            \begin{figure}[!h] 
                \centering
                \includegraphics[width=0.95\textwidth]{img/logo}
                % \caption[Algunas aplicaciones de la visión artificial]{Algunas aplicaciones de la visión artificial. Fuente: \cite{Szeliski2011} }
            \end{figure}
        \end{column}
    \end{columns}
    
\end{frame}
\begin{frame}{Historia}
    \begin{columns}
        \begin{column}{0.6\textwidth}
            A partir de la década de los 2010, Python gana tracción por su facilidad de uso y el 
            desarrollo de librerías y módulos para el desarrollo de algoritmos de inteligencia artificial 
            y deep learning. 

            En la actualidad, Python es considerado como un lenguaje fundamental en AI y Data Science.

        \end{column}
        \begin{column}{0.4\textwidth}
            \begin{figure}[!h] 
                \centering
                \includegraphics[width=0.95\textwidth]{img/logo}
                % \caption[Algunas aplicaciones de la visión artificial]{Algunas aplicaciones de la visión artificial. Fuente: \cite{Szeliski2011} }
            \end{figure}
        \end{column}
    \end{columns}
    
\end{frame}

\begin{frame}{Características Principales}
    \begin{columns}
        \begin{column}{0.6\textwidth}
            Python ha ganado tracción por las siguientes razones:
            \begin{itemize}
                \item Sintaxis simple y limpia.
                \item Programación multiparadigma.
                \item Tipado dinámico.
                \item Lenguaje interpretado.
                \item Open Source.
                \item Modo interactivo.
            \end{itemize}
        \end{column}
        \begin{column}{0.4\textwidth}
            \begin{figure}[!h] 
                \centering
                \includegraphics[width=1.35\textwidth]{img/logo2}
            \end{figure}
        \end{column}
    \end{columns}
    
\end{frame}

\begin{frame}{¿Dónde se usa Python?}
    \begin{columns}
        \begin{column}{0.6\textwidth}
            Python es ampliamente usado en diversos campos:
            \begin{itemize}
                \item Desarrollo web.
                \item Investigación científica y numérica.
                \item Educación.
                \item Desarrollo de videojuegos.
                \item Interfaces gráficas.
                \item Automatización.
            \end{itemize}
        \end{column}
        \begin{column}{0.4\textwidth}
            \begin{figure}[!h] 
                \centering
                \includegraphics[width=1.35\textwidth]{img/logo2}
            \end{figure}
        \end{column}
    \end{columns}
    
\end{frame}
%%%%
\begin{frame}[fragile]{Comparación Python vs C++}
    C:
    \begin{lstlisting}
        #include <stdio.h>
        int main(int argc, char **argv)
        {
            printf("hola mundo");
            return 0;
        }
    \end{lstlisting}
    Python:
    \begin{lstlisting}
        print("hola mundo")
    \end{lstlisting}
\end{frame}

\subsection{Lenguajes compilados vs interpretados}
\begin{frame}{Lenguajes Compilados}
    En un lenguaje compilado, un archivo de \alert{código fuente} 
    es procesado por un programa especial llamado \alert{compilador} que 
    se encarga de transformar el código en un programa \alert{ejecutable}.
    \begin{itemize}
        \item C++.
        \item C.
        \item Java.
        \item Rust.
    \end{itemize}
    La ventaja es que el compilador puede optimizar el código fuente y 
    generar ejecutables muy eficientes.
\end{frame}

\begin{frame}{Lenguajes Interpretados}
    En un lenguaje interpretado, no existe un compilador que convierte código fuente
    en ejecutables. Se tiene un programa \alert{intérprete} que ejecuta los comandos dados
    uno por uno de forma \alert{inmediata}.
    \begin{itemize}
        \item Python.
        \item Bash.
        \item Php.
    \end{itemize}
    La ventaja es que se puede iterar en el desarrollo de forma más rápida.
\end{frame}
\subsection{Instalación en Windows}
\begin{frame}{Descarga}
    El entorno de ejecución de Python se puede descargar como un \alert{instalador} en Windows. Este instalador 
    cuenta con todo lo necesario para comenzar a desarrollar.

    El instalador se puede descargar de la página \href{https://www.python.org/downloads/windows/}{python.org}

    En Linux, Python viene instalado \alert{por defecto}.

\end{frame}

\subsection{El intérprete de python}

\subsection{Operadores aritméticos}
\subsection{Variables 1}

\section{Sintaxis y características del lenguaje}
\subsection{Hola mundo en python}
\subsection{Tipos de datos}
\subsection{Operadores aritméticos y lógicos}
\subsection{Variables 2}
\subsection{Contenedores y colecciones}
\subsection{Ejecución de scripts}
\subsection{Entrada y salida de datos}


\section{Programación con Python 3}
\subsection{Funciones}
\subsection{Control de flujo}
\subsection{Argumentos desde línea de comandos}
\subsection{Manejo de archivos}

\section{Introducción a la programación orientada a objetos}
\subsection{Clases y objetos}
\subsection{Modularización y encapsulamiento}
\subsection{Herencia y polimorfismo}
\subsection{Módulos de python}


\section{Tópicos avanzados}
\subsection{Web Scrapping*}
\subsection{Interfaces Gráficas*}
\subsection{Videojuegos*}


\section*{Pruebas}
\begin{frame}{Puntajes y errores en el conjunto de prueba}
    \begin{columns}
        \begin{column}{0.4\textwidth}
            \small{Los puntajes sobre el conjunto de prueba definen el rendimiento y la capacidad de generalización de la 
            red sobre datos nunca antes vistos. Se usan las siguientes métricas:}
    
            \begin{itemize}
                \item \alert{MSE}: Error Cuadrático Medio.
                \item \alert{MAE}: Error Absoluto Medio.
                \item \alert{$R^2$}: Coeficiente de Determinación.
            \end{itemize}
        \end{column}
        \begin{column}{0.6\textwidth}
            \begin{table}[!h]
                \centering
                \resizebox{0.95\textwidth}{!}{%
                \begin{tabular}{@{}|c|c|c|c|@{}}
                \toprule
                \textbf{Modelo} & \textbf{MSE} & \textbf{MAE} & $\mathbf{R^2}$ \\ \midrule
                Tradicional & 0.0254 & 0.0976 & 0.9278 \\ \midrule
                Convolucional & 0.0160 & 0.0872 & 0.9484 \\ \bottomrule
                \end{tabular}%
                }
                \caption[Evaluación de puntajes sobre el conjunto de prueba.]{Evaluación de puntajes sobre el conjunto de prueba. Fuente: Elaboración propia.}
                \label{tbl:testscores}
            \end{table}
        \end{column}
    \end{columns}
\end{frame}


{\setbeamercolor{palette primary}{fg=black, bg=yellow}
\begin{frame}[standout]
  Preguntas?
\end{frame}
}

\appendix

% \begin{frame}[fragile]{Backup slides}
%   Sometimes, it is useful to add slides at the end of your presentation to
%   refer to during audience questions.

%   The best way to do this is to include the \verb|appendixnumberbeamer|
%   package in your preamble and call \verb|\appendix| before your backup slides.

%   \themename will automatically turn off slide numbering and progress bars for
%   slides in the appendix.
% \end{frame}

\begin{frame}[allowframebreaks]{Referencias}
.
\end{frame}

\end{document}
